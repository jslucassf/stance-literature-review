\documentclass[12pt, sigconf]{acmart}

\usepackage{booktabs} % For formal tables

% Copyright
%\setcopyright{none}
%\setcopyright{acmcopyright}
%\setcopyright{acmlicensed}
\setcopyright{rightsretained}
%\setcopyright{usgov}
%\setcopyright{usgovmixed}
%\setcopyright{cagov}
%\setcopyright{cagovmixed}

\begin{document}
\title{Detecção de Posicionamento em Dados Textuais}
\subtitle{O estado da arte e considerações}


\author{José Lucas Silva Freitas}
\affiliation{%
  \institution{Universidade Federal de Campina Grande}
  \city{Campina Grande}
  \state{Paraíba}
}
\email{jslucassf@gmail.com}

% The default list of authors is too long for headers.
\renewcommand{\shortauthors}{Lucas Freitas et al.}


\begin{abstract}

\quad  The task of stance detection in textual data can be understood as
given some text, automatically identifiyng wheter the author is in favor of,
against or even neutral about a certain target. The target may be an idea,
a person, a company or even a product. For instance, automatically detecting
a politician's positioning towards a certain government policy
on his speeches. The text to be analysed can even come from a normal person 
posting in social media platforms like Twitter or Facebook.

\quad  In this work, many papers that address this problem were studied. Our goal
being to better understand the nature of the problem, listing the techniques
that were already used in dealing with it and identifiyng possible open
areas that may be explored in future works.

\quad  The solution to this problem may be really important for various systems
based on interaction with users, based on text written by humans. One
of the main applications that were mentioned in the works that we studied
was the detection of fake news, mostly due to the troublesome social and 
political moment, in which we see ourselves nowadays.
\end{abstract}
\keywords{Texto, Posicionamento, Linguagem, notícias}
\maketitle
\section{Introdução}

\section{Mineração de Textos}

\section{Fake News e aplicações}

\section{Detecção de Posicionamento x Detecção de Sentimento}

\section{Abordagens baseadas apenas no texto}

\section{Regras de Classificação}

\section{Utilizando Outras informações Relevantes}

\subsection{Informações temporais}

\subsection{Estrutura das Conversas em Redes Sociais}

\section{Deep Learning}

\section{Conclusões}


\bibliographystyle{ACM-Reference-Format}
\bibliography{sample-bibliography}

\end{document}
