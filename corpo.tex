\section{Introdução}
\quad Politicamente e socialmente, vivemos tempos conturbados. A cada dia
novos temas polêmicos são trazidos para discussão da sociedade e cada
vez mais, as pessoas se sentem no dever de se posicionarem. Entender
tais posicionamentos pode ser muito útil, para entender o espírito do
posicionamento da sociedade em relação aos temas mais importantes.
\quad Além disto, entender o posicionamento das pessoas pode ter
várias outras aplicações, como desenvolvimento de aplicações específicas,
e mais recentemente ajudar na detecção de notícias falsas, as 
\emph{fake news} \cite{mrowcastance}.

\quad Em reconhecimento da importância deste problema, vários trabalhos
utilizando diversas técnicas para detecção automática de posicionamento
foram publicados. A associação de linguística computacional (ACL)
organiza anualmente o \emph{SemEval},
\footnote{https://aclweb.org/aclwiki/SemEval\_Portal} um workshop com vários
desafios na área da análise de semântica computacional, dentre estes
desafios, algumas tarefas relacionadas a detecção de posicionamento já
foram propostas. Estudar a contribuição de todos estes trabalhos e os
métodos por eles utilizados é muito importante pois possibilita o
entendimento do estado da arte e a identificação de possíveis
territórios de pesquisa ainda não explorados.

\quad Iniciamos com uma visão geral da área da mineração de textos. Explorando
o estado da arte no processamento de textos, mostramos de uma forma geral,
como são realizadas as tarefas de processamento de texto. Então falamos um
pouco mais sobre a aplicação em alguns contextos, especialmente das fake news.
Prosseguimos para esclarecer as diferenças entre duas tarefas da análise 
semântica, a detecção de posicionamento e a detecção de sentimento. Nas
próximas sessões, exploramos abordagens que usam os textos, regras, e outras
informações relevantes, como as informações temporais e a estrutura das
conversas (quando analisamos textos produzidos em redes sociais).

\section{Mineração de Texto}

\quad A Mineração de texto é uma área multidisciplinar que envolve recuperação da
informação, análise de dados, visualização, classificação, aprendizado de máquina
e etc. Quando tentamos detectar o posicionamento dos autores em um texto, estamos
implicitamente buscando extrair informações semânticas importantes e não triviais
e isto é exatamente como podemos definir mineração de texto, o processo de busca
e extração de informações importantes em texto. Desta forma, se torna muito
importante entender inicialmente o estado atual do nosso entendimento em técnicas
e conceitos de mineração de texto, pois os mesmos serão aplicados no contexto do
nosso problema.

\quad O processamento de informações armazenadas de forma não estruturada é
essencial para negócios em geral. Alguns estudos [Referência] já demonstraram que
cerca de 80\% de todas as informações potencialmente úteis para negócios está
armazenada em formato não estruturado. Como texto sempre foi a forma mais natural
de transmissão e armazenamento de conhecimento humano, a maior parte destes dados
não estruturados está em formato textual, portanto o desenvolvimento de técnicas
de processamento de texto possui um enorme potencial de geração de valor, tanto
para empresas quanto para usuários finais das novas tecnologias que poderão ser
desenvolvidas.

\subsection{Um Framework Conceitual}

[Referência] apresenta um framework conceitual para tarefas de mineração de texto
que consiste em dois componentes:
\begin{enumerate}
    \item \emph{Refinamento de texto}
    \item \emph{Destilação de conhecimento}
\end{enumerate}

\quad A etapa de \textbf{refinamento de texto}, consiste em uma etapa inicial, onde ocorre a
transformação dos documentos de texto em um tipo de representação intermediária de
mais fácil processamento, alguns exemplos são organizações relacionais dos
documentos, grafos de conceitos ou até contagem de palavras. O formato de
representação intermediária pode ser baseado em documentos, onde cada entidade nos
dados representa um documento ou em conceitos, onde cada entidade representa um
conceito ou objeto de interesse no estudo.

\quad Já na etapa de \textbf{destilação de conhecimento}, a representação 
intermediária dos documentos, produzida na fase anterior, é finalmente processada,
possibilitando a dedução dos padrões e informações desejadas. Nesta etapa, várias
técnicas de processamento são utilizadas, como visualização de documentos, análise
de texto.

\quad A ideia principal na abordagem baseada em visualização, é o agrupamento de
documentos com base em suas similaridades e apresentar os grupos definidos fazendo
uso de algum tipo de representação gráfica. A interpretação da semântica dos grupos
ainda é uma tarefa a ser realizada por humanos.

\quad Análise de texto é na verdade apenas um termo genérico que engloba muitas
abordagens diferentes. Processamento de Linguagem Natural (NLP), Sumarização,
extração de informação e etc. Aqui destacamos o crescente uso de técnicas de
aprendizado de máquina, que vem se popularizando na pesquisa atual, devido ao
avanço no poder de processamento dos computadores atuais.

\subsection{Problemas em Aberto}

\quad Apesar dos avanços nos últimos anos, ainda enfrentamos alguns desafios na área.
O concebimento de um formato intermediário eficiente para o processamento, o
desenvolvimento de técnicas e algoritmos capazes de produzirem soluções mais
independentes da linguagem e a exploração de formas de fazer uso de informações
relativas ao domínio específico dos problemas, são exemplos de aspectos que
necessitam de novas contribuições para no futuro.

\section{Fake News e aplicações}

\quad Sem dúvidas, a maior aplicação das técnicas de detecção de posicionamento
atualmente está na verificação das chamadas \textit{Fake News}. As notícias falsas
tem causado muitas preocupações, principalmente em cenários eleitorais onde elas
são suspeitas até mesmo de influenciar o resultado de campanhas presidenciais.
\footnote{https://www.washingtonpost.com/news/the-fix/wp/2018/04/03/a-new-study-suggests-fake-news-might-have-won-donald-trump-the-2016-election/}

\quad Desta forma, existe o interesse em identificar a veracidade de notícias
veiculadas na internet, mas como uma quantidade muito grande de informações circulam
a cada dia, a automatização deste processo é desejada.

\quad Vários estudos [Referência] sugerem que detectar o posicionamento dos autores
das notícias e de outros portais pode ser um primeiro passo essencial para verificar
a veracidade das mesmas. Profissionais da academia e várias empresas ao redor do
mundo realizam o \textit{Fake News Challenge
\footnote{http://www.fakenewschallenge.org/}}, um desafio que objetiva abordar este
problema, e também sugere a detecção de posicionamento como primeiro passo para a
verificação de veracidade de notícias.

\quad Entretanto as aplicações podem ser ainda maiores. Alguns trabalhos [Referência]
já fizeram uso de estruturas similares para realizar não só a verificação das notícias
oficialmente publicadas, mas também para boatos que circulam as redes sociais.

\section{Detecção de Posicionamento x Detecção de Sentimento}

\quad A detecção de sentimento em texto, é um campo com vários trabalhos publicados,
porém é importante ressaltar que como citado por [Referência], detecção de
posicionamento é uma tarefa relacionada mas diferente da análise de sentimento.
Em tarefas de análise de sentimento, buscamos determinar se um texto é positivo,
negativo ou neutro. Geralmente extraímos também o alvo daquela opinião. Entretanto
para detecção de posicionamento, buscamos identificar a concordância com um assunto
pré-estabelecido. O assunto alvo da opinião pode não ser mencionado no texto, ou até
não ser o sujeito do texto.

\quad Estas são tarefas distintas porém são relacionadas. Em [Referência] por exemplo,
regras de sentimento são utilizadas para auxiliar na detecção de posicionamento. Uma
pergunta permanece em aberto entretanto, qual o nível de correlação entre as
duas tarefas?

\section{Abordagens baseadas apenas no texto}

\quad Durante a primeira etapa no framework que foi citado, tomamos decisões sobre o
formato dos dados que serão processados, ou seja, decidimos sobre a representação
intermediária. Uma decisão que pode ser feita, é se usaremos apenas os dados
textuais dos documentos ou se faremos uso de outros tipos de informação que sejam
releventes. Em [Referência], os autores realizam a classificação de \textit{tweets}
realizando o processamento apenas no conteúdo textual dos mesmos. A abordagem faz uso
de diferentes classes de n-gramas como representação intermediária e gera modelos
preditivos utilizando um algoritmo de classificação (SVM).

\section{Regras de Classificação}

\quad O trabalho citado anteriormente [Referência] também faz uso de regras de
sentimento em uma primeira etapa do processo de classificação. Regras de
sentimento. Estas, fazem uso da presença de alguns tipos de n-gramas no texto que
podem ser indicativos do sentimento do autor para com os alvos da opinião.
Os n-gramas alvo, são aqueles usados para representar direta ou indiretamente os
alvos da questão. Por exemplo, nomes próprios. Os n-gramas chave, são os que servem
para deixar explícito a opinião do autor. Por exemplo \textit{hashtags} ou termos
como "readyforhillary". 

\quad Cada n-grama é então separado em dois tipos, favoráveis e contrários em relação
ao alvo principal. Desta forma, criam-se pares ngrama-tipo, que compõem as regras de
sentimento utilizadas para classificar os tweets. As regras são no formato:

\begin{itemize}
    \item Presença de n-grama chave favorável e ausência de n-grama chave contrário
    implica em posicionamento Favorável
    \item Presença de n-grama chave contrário e ausência de n-grama chave favorável
    implica em posicionamento Contrário 
\end{itemize}

\quad O conjunto das regras geradas, são então utilizadas para classificar o
posicionamento dos tweets.

\quad Este é um processo semi automático pois necessita de supervisão humana para a
anotação dos n-gramas, e no trabalho citado é utilizado como um primeiro passo para
rotulação um conjunto de dados que será útil para treinar um modelo preditivo.

\quad Ainda sobre análise de sentimento, esta abordagem se apoia na suposição de
que o posicionamento pode ser expresso na forma de sentimentos positivos ou
negativos em relação aos alvos do texto.

\section{Utilizando Outras Informações Contextuais}

\quad Outros trabalhos abordam o problema de detecção de posicionamento
fazendo uso de outras informações além dos textos sendo analisados. [Referência]
se aproveita da estrutura das conversas em redes sociais enquanto [Referência]
considera informações temporais das postagens no Twitter.

\subsection{Estrutura das Conversas em Redes Sociais}

\quad As redes sociais possibilitam uma estruturação nas conversas, 
que segundo [Referência], pode influenciar o poder de classificação de nossos
métodos. A intuição, é que após um rumor ser compartilhado, as postagens resposta ao
texto original podem auxiliar a verificação da informação pois seus
posicionamentos podem representar contra argumentações ou apoios à ideia
inicial. Outro trabalho que também faz uso de informações relacionadas a
estrutura das conversas, é [Referência]. Nele os autores ainda chamam
atenção sobre o fato de que as abordagens baseadas em regras, que foram
citadas anteriormente, possuam a falha de não poderem ser generalizáveis
para diferentes contextos.

\subsection{Informações temporais}

\quad Em [Referência] os autores estudam mais uma vez dados de redes sociais
e levantam a hipótese de que a utilização de dados temporais afeta
positivamente a precisão dos métodos de detecção. Uma ideia importante é que
a ocorrência de um tweet influencia a taxa em que futuros tweets serão
publicados.

\section{Conclusões}

\quad Este artigo buscou analisar a literatura com o objetivo de organizar
o conhecimento relacionado ao estado da arte na pesquisa de detecção de
posicionamento em dados textuais, suas aplicações e desafios. Vimos que
esta é uma tarefa com aplicações especiais na verificação de veracidade em
notícias e rumores.