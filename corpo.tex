\section{Introdução}
\quad Politicamente e socialmente, vivemos tempos conturbados. A cada dia
novos temas polêmicos são trazidos para discussão da sociedade e cada
vez mais, as pessoas se sentem no dever de se posicionarem. Entender
tais posicionamentos pode ser muito útil, para entender o espírito do
posicionamento da sociedade em relação aos temas mais importantes.
\quad Além disto, entender o posicionamento das pessoas pode ter
várias outras aplicações, como desenvolvimento de aplicações específicas,
e mais recentemente ajudar na detecção de notícias falsas, as 
\emph{fake news} [Referência].

\quad Em reconhecimento da importância deste problema, vários trabalhos
utilizando diversas técnicas para detecção automática de posicionamento
foram publicados. A associação de linguística computacional (ACL)
organiza anualmente o \emph{SemEval},
\footnote{https://aclweb.org/aclwiki/SemEval\_Portal} um workshop com vários
desafios na área da análise de semântica computacional, dentre estes
desafios, algumas tarefas relacionadas a detecção de posicionamento já
foram propostas. Estudar a contribuição de todos estes trabalhos e os
métodos por eles utilizados é muito importante pois possibilita o
entendimento do estado da arte e a identificação de possíveis
territórios de pesquisa ainda não explorados.

\quad Iniciamos com uma visão geral da área da mineração de textos. Explorando
o estado da arte no processamento de textos, mostramos de uma forma geral,
como são realizadas as tarefas de processamento de texto. Então falamos um
pouco mais sobre a aplicação no contexto especialmente das fake news.
Prosseguimos para esclarecer as diferenças entre duas tarefas da análise 
semântica, a detecção de posicionamento e a detecção de sentimento. Nas
próximas sessões, exploramos abordagens que usam os textos, regras, e outras
informações relevantes, como as informações temporais e a estrutura das
conversas (quando analisamos textos produzidos em redes sociais).

\section{Mineração de Textos}

\section{Fake News e aplicações}

\section{Detecção de Posicionamento x Detecção de Sentimento}

\section{Abordagens baseadas apenas no texto}

\section{Regras de Classificação}

\section{Utilizando Outras informações Relevantes}

\subsection{Informações temporais}

\subsection{Estrutura das Conversas em Redes Sociais}

\section{Conclusões}
